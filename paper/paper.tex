%\documentclass[10pt,twocolumn]{article}
\documentclass[10pt, conference]{IEEEtran}
%\IEEEoverridecommandlockouts
%\documentclass{article}
%\documentclass[12pt]{IEEEtran}
%\documentclass{elsart5p}

% DFRWS Call Information:
% http://www.dfrws.org/2009/cfp.shtml
% Important Dates
% Submission deadline: March 16, 2009 (any time zone)
% Author notification: April 28, 2009
% Final draft due: May 19, 2009
% Pre-conference Workshops: August 16, 2009
% Conference dates: August 17-19, 2009

% Publication Criteria
% Research papers must be original contributions, not substantially
% duplicate previous work, and must not be under simultaneous
% publication review elsewhere. The review process will be
% ``double-blind'' (the reviewers will not know who the authors are, and
% the authors will not know who the reviewers are). Therefore, the
% version submitted for review should not contain the names or
% affiliations of the authors. When referring to their own previous
% work, authors should use the third person instead of the first person
% (i.e. ``Smith and Jones [2] previously determined...'' instead of ``We
% [2] previously determined..''). Authors are expected to present their
% work in person at the workshop and must have at least one registration
% per paper in order to be included in the proceedings.


% Make sure we know its a draft for now
\usepackage{draftwatermark}
\SetWatermarkFontSize{3cm}
\usepackage{fancyvrb}
%\documentclass[12pt]{article}
\usepackage{epsfig}
\usepackage[hypertexnames=false,bookmarksopenlevel=1,bookmarksopen,bookmarksnumbered,colorlinks,plainpages=false,linktocpage,linkcolor=black,citecolor=black,filecolor=black,urlcolor=black]{hyperref}
%\usepackage{natbib}
%\pagestyle{empty}
%\psdraft
%\baselineskip=20pt %Sets line spacing to 1 unit
%\bibliographystyle{elsart-num-names}
\bibliographystyle{IEEEtran}
\usepackage{cite}
\newenvironment{definition}[1][Definition]{\begin{trivlist}
\item[\hskip \labelsep {\bfseries #1:}]}{\end{trivlist}}

%\begin{frontmatter}
\begin{document}
\title{Extending the Advanced Forensic Format to accommodate Multiple
  Data Sources, Logical Evidence and Forensic Workflow}
\author{Blinded for review}
%\author{M. I. Cohen\footnote{scudette@gmail.com}, Simson Garfinkel, and Bradley Schatz}
%\author{M. I. Cohen}
%\address{M. I. Cohen is a Data specialist with the Australian Federal Police, Brisbane, Australia}
%\baselineskip=20pt %Sets line spacing to 1 unit
%\end{frontmatter}
%\thanks{M. I. Cohen is a Data specialist with the Australian Federal Police, Brisbane, Australia}
\maketitle

\begin{abstract}
Forensics analysis requires the acquisition and management of many
different types of evidence material, including individual disk
drives, RAID sets, network packets, memory images, and extracted files. Often the same evidence is
reviewed by several different tools or examiners in different
locations. We propose a backwards-compatible evolutionary redesign of the Advanced
Forensic Format an open, extensible
file format for storing of evidence and interchanging analysis results
among different tools. The Forensic Interchange Format (FIF) was
designed to be simple to implement, relying on the well supported Zip
specifications for bit level file access.
\end{abstract}

\section{Introduction}

Unlike other kinds of evidence, digital evidence can be instantly lost
or corrupted if technical measures are not immediately taken to
preserve its existence and integrity. As a result, preserving digital
evidence is an important part of most digital
investigations\cite{carrier:event-based}.

In recent years there has been a steady and growing interest in the
actual file formats and containers used to store this digital
evidence. Early practitioners created exact bit-for-bit copies (``mirror disks'') employed
proprietary software systems (e.g. \cite{safeback,ilook,encase}) for
making and authenticating ``images'' of digital evidence. PyFlag\cite{pyflag}
introduced a ``seekable gzip'' format that allowed disk images to be
stored in a form that was compressed but allowed the random-access to
evidence data necessary for forensic analysis. The Advanced Forensic
Format (AFF) expanded on this idea with a forensic file format that
allowed both data and arbitrary metadata to be stored in a single
digital archive\cite{garfinkel:aff}. Recently the open source AFF
implementation library (AFFLIB) was expanded to support both
encryption and digital signatures of digital evidence\cite{garfinkel:affcrypto}.

This paper extends the previous work by adapting AFF to the multiple
heterogeneous data types that might arise in a modern digital
investigation, including data from multiple data storage devices, new
data types (including network packets and memory images), extracted
logical evidence, and forensic workflow. We do this in a manner that
is at once upwards compatible with AFF while adopting the AFF
bit-level specification so that AFF files can be processed using
standard non-forensic tools.  We call the new system AFF4, and use the
phrase AFF1 to refer to the legacy system developed by
Garfinkel.\footnote{Although Garfinkel never changed the AFF bit-level
specification, Garfinkel released AFFLIB implementations with major
version numbers 1, 2 and 3. We therefore call our system AFF4 to avoid confusion.}

\begin{figure}
\fbox{
\begin{minipage}{.95\linewidth}
\begin{itemize}
\item The ability to store evidence from one or more physical
  devices.
\item Schema allowing evidence files to contain memory
  images, network packet intercepts, Netflow data, and extracted
  files.
\item Direct encoding of RAID configuration, allowing RAID volumes
  to be automatically reassembled from evidence files made of the
  physical RAID devices.
\item A revised compression system that allows individual chunks
  within a 16MB segment to be decompressed without decompressing the
  entire segment. 
\item Support for Global Distributed Evidence Management.
\end{itemize}
\end{minipage}}
\caption{Specific design goals for the next generation forensic file
  format.}
\end{figure}
\section{The Need for an Improved Forensic Format}

AFF1's flexibility came from the fact that forensic data and metadata
is stored as arbitrary name/value pairs called \emph{segments}. For
example, the first 16MB of a disk image is stored in a segment called
\texttt{page0}, the second 16MB in a segment called \texttt{page1},
\emph{etc.} Because of this flexibility, it was relatively easy for
Garfinkel to extend AFF1 to support encryption, digital signatures, and
the storage of new kinds of metadata such as chain-of-custody
information\cite{garfinkel:affcrypto}. 

\subsection{AFF Limitations}
After imaging literally thousands of devices using Garfinkel's AFF1
tools and adding support for AFF1 to numerous open source programs, we
observed numerous problems in the underlying standard
and Garfinkel's AFFLIB implementation:

\begin{itemize}
\item While AFF1's design stores a single disk image in each evidence
  file, modern digital investigations typically involve many seized
  computers or pieces of media. 
\item AFF1 has no provision for storing memory images or intercepted
  network packets.
\item AFF1 has no provisions for storing extracted files that is
  analogous to the EnCase ``Logical Evidence File'' (L01) format, or
  for linking evidence to web pages.
\item AFF1's encryption system leaks information about the contents of
  an evidence file because segment names are not encrypted.
\item AFF1's default compression page size of 16MB can impose significant overhead
  when accessing NTFS Master File Tables (MFT), as these structures
  tend to be highly fragmented on systems that have seen significant
  use. 
\item Although the AFF1 specification calls for a ``table of contents'' similar
  to the ZIP\cite{zip-format} ``central directory'' that is stored at the end of AFF
  files, Garfinkel never implemented this directory in the publicly
  released AFF1 implementation, AFFLIB. As a result, every header of every segment in an
  AFF file needs to be read when a file is opened. It practice this
  can take 10--30 seconds the first time a large AFF file is
  opened. (Once the file is opened, the sectors corresponding to the
  segment headers are stored in the computer's disk cache.)
\item AFF1's bit-level specification is essentially a simple container
  file specification. Given the that there are other container file
  specifications that are much more widely supported with both
  developer and end-user tools, it seemed reasonable to migrate AFF
  from its home-grown format to one of the existing standards. 
\end{itemize}

\subsection{Global Distributed Evidence Management}
While Garfinkel designed AFF for use on a single machine that could
both image evidence and perform analysis, the main use of AFF in
practice has been in distributed environments in which imaging and
analysis takes place in multiple locations and is performed by
multiple individuals.

Managing evidence in a globally distributed system requires the use of
globally unique identifiers. The original AFF specification assigned
each piece of evidence a unique 128-bit identifier called a GID but did
not make it clear when this identifier should be changed and when it
should remain the same. 

In practice, however, global distributed evidence management requires
more than simply tracking the movement of disk images: it requires
approaches for sharing evidence to multiple disconnected
evidence, allowing offline work, and then seamlessly recombining the
work products of the analysts in a third security domain.

Consider the typical usage scenario depicted in Figure \ref{usage}, of
an file which contains a disk image. This volume is distributed to two
independent analysts, Alice and Bob. Alice may find and extract
individual files, while Bob may correlate information in the evidence
file with other data that is available on departmental
servers. Although in some environments Alice and Bob may be able to
work on a shared file that is located on a server, in other
environments there will not be sufficient bandwidth or
connectivity. Instead, each analyst will be required to store the
information in their own evidence file; these files will then be
recombined at a later point in time.

In this case they can
each create a new volume which extends to original volume and save
their analysis on this new volume. Now they only need to share this
new volume with other analysts who also have a copy of the old volume
to interchange their findings. 

This is made possible because each volume is independent of one
another, but can still be recombined into a bigger evidence set. 

\begin{figure}[tb]
  \begin{center}
  \mbox{\epsfxsize=0.6\columnwidth \epsffile{usage1.eps}}
  \caption{A typical usage scenario. Both Alice and Bob receive an AFF
  volume but work independently. Rather than modifying the volume,
  they each create their own local volumes and save their results into
  those files. They can now exchange the smaller new volumes and
  effectively merge their results into the same AFF set when they are finished.}
  \label{usage}
  \end{center}
\end{figure}

\subsection{Migration from AFF1 to AFF4}

We have addressed these issues in three steps. First, we have extended
the AFF schema to support multiple forensic images (which are now
called \emph{streams}) in a single evidence file. Second, we replaced
AFF's ``bit-level'' format with an improved format that is more
flexible and has better existing support. Third, we are adopting
AFFLIB to implement the new schema and file format, which will allow
existing AFF-aware programs to access files written in either the old
or the new format without modification.

We call the result of our work effort AFF4.

\section{Introducing AFF4}


\subsection{New Concepts and Terminology}

This section discusses the AFF4 terminology. 

\begin{itemize}
\item \emph{An Evidence Volume} is a single file with a \texttt{.aff}
  extension. 
\item \emph{An Evidence Set} is a collection of one or more Evidence
  Volumes. Evidence Sets can also be stored in object storage system (such as
  Amazon S3), or a hybrid---for example, metadata stored in an SQL
  database with physical data stored in large files in a file
  system. 
\item \emph{An Evidence Set UUID} is a unique identifier generated
  according to RFC4122\cite{rfc4122} for each new Evidence
  Set. Multiple Evidence Volumes that make up the same Evidence Set
  should all have the same UUID. When an Evidence Set is opened all its volumes are
  loaded, their UUIDs are checked, and all of their contents are
  logically merged so as to be visible as if they were all found in the same volume.
\item \emph{A segment} is a single unit of data written to volume. AFF4
  segments have a \emph{segment name}, a \emph{segment timestamp}, and
  a \emph{segment contents}.

\item \emph{A stream} is a collection of segments that can be
  concatenated together to form a single data object.

\item \emph{An image stream} is a stream that is dedicated to storing evidence from a single
  source of data.  For example, a stream may contain a disk image or a network
  capture. Whereas AFF could only support a single stream in an evidence
  file, AFF4 allows any number of streams to be stored. Each stream in
  the archive is  given a unique identifier.

\item \emph{A bevy} is a specific segment that is used to hold
  forensic image data, such as disk sectors, memory pages, or network
  packets. AFF1 used the term ``pages'' to describe bevys, but this
  term caused confusion when AFF was used to hold memory images, hence the
  new term. 

\item \emph{A segment name} is the name under which a segment is
  stored within the archive. AFF4 segments have two-part names that
  typically contain a \emph{stream name} and \emph{bevy number}. For
  example the segment named \texttt{default/000001} refers to the stream
  called \texttt{default} and has a chunk name of \texttt{000001}. All segment
  names are encoded using UTF8. AFFLIB is being modified to allow any
  stream to be opened using the \texttt{af\_open} function; if a
  stream name is not specified, AFF4 implementations will open the
  stream named ``default.''

\item \emph{The segment timestamp} is a timestamp for the segment
  recorded in GMT.

\item \emph{Attributes} are a list of key/value pairs stored in each
stream's ``properties'' segment (e.g. ``stream/properties''). The
``properties'' file is a text file which stores attributes one on each
line. Attribute lines consist of the key, immediately followed by
``='' for text values or ``:'' for binary values. This is immediately
followed by the encoded value. Values containing text are encoded
using UTF8, while values containing binary data are encoded using
Base64. There may be multiple values for each attribute; in these
cases the order is preserved
by implementations. Implements may present different keys may be
presented in any order, however.
\vspace{1ex}
Each AFF4 archive must also have a top-level set of attributes---that
is, there is always a segment named ``properties.'' These volume
attributes are used to specify a Unique Identifier (UUID). (AFF1
stored each property in its own named segment. This created
significant overhead in many situations.)

\item \emph{A Target Reference} is a way of referencing objects. This is used
in a number of places which will be highlighted below. Using
references it is possible for the FIF file to refer to objects
embedded within itself, or within other FIF files. Its is also
possible to refer to external flat files. A number of usage scenarios
are described in Section \ref{usage_scenarios}. 
\end{itemize}

The Target Reference is an important AFF4 innovation: this mechanism
gives AFF4 files a way for describing evidence that is stored in
different evidence volumes, in the host computer's file system, or on
the Internet. We have defined the following target reference schema:

\label{target_reference}

\begin{itemize}
\item A reference in the format {\em aff://UUID/stream/segment} refers
to an external Evidence file set (identified by its UUID). If the segment
is missing (i.e. the reference ends with a ``/''), the reference is
for the entire stream. We do not specify a way of actually resolving
the UUID into a physical locations. This can achieved by
implementations using many methods, such as Active Directory, LDAP,
DNS, flat files or SQL databases.

\item A reference starting with {\em file://} refers to an external
file. The external file reference must be sanitised (to remove
.. directory traversal) and rooted at the current directory. A
filename of ``.'' refers to the current FIF volume.

\item A reference starting with {\em http://} refers to a HTTP
object which will be fetched. Support for this is optional.

\item A references ending with a ``/'' signifies a reference to a
stream in the current FIF file. The stream will be opened in its
entirety. If the reference does not end with a ``/'', the reference is
for a single segment.
\end{itemize}

\subsection{Streams}
As discussed above, individual segments in an AFF4 evidence set can be
combined into a single logical object called a \emph{stream}. AFF4
supports an arbitrary number of streams per volume. The AFFLIB library
allows the segments in a stream to be operated on as if they were a
single file inside a file system by supporting the traditional
POSIX-like functionality of \texttt{open()}, \texttt{seek()} and
\texttt{read()}.

AFFLIB also supports the \emph{writing} of streams, a feature that is
used by Garfinkel's \texttt{aimage} disk imaging program. In general
arbitrary writing to streams can be expensive, especially when writing
is combined with seek operations. The current \texttt{aimage} program
will create AFF files with gaps in them during its error recovery
process (when \texttt{aimage} seeks to the end of a disk and reads it
backwards in an attempt to get around media failures). We are working
on a rewrite of \texttt{aimage} which performs imaging and recovery
on bevy boundaries, rather than using the current arbitrary
``high-water-mark'' and ``low-water-mark.''

All streams within an Evidence Set \emph{must} have a properties segment. The
properties segment describes what the stream contains and determines
how the stream is processed. 


\subsection{ZIP64}
For AFF4, we have changed AFF1's original container file format to
ZIP64\cite{zipspecs}. There are many reasons for this decision:

\begin{itemize}
\item There is already wide support for the ZIP and ZIP64 formats. By
  migrating to these formats, we can take advantage of the rich number
  of user and developer tools already available.
\item ZIP64 includes a CRC32 and timestamp for each segment. We
  believe that the additional integrity checks and timestamps will be
  useful in a forensic environment.
\item ZIP64 already includes support for the ``table of contents''
  envisioned but never implemented by Garfinkel; ZIP64 calls this
  table the ``Central Directory.'' It is written at the end of the
  ZIP64 volume.
\end{itemize}

Figure~\ref{zip_structure} shows the basic structure of a Zip
archive. As can be seen, the archive consists of a {\em Central Directory} (CD)
located at the end of the archive. The CD is a list of pointers to
individual {\em File header} structures located within the body of the
archive. Headers are then followed by the file data, after it has been
compressed by the appropriate compression method (as specified in the
header). Each archived file is optionally followed by a {\em Data
Descriptor} describing the length and CRC of the archived file. Using
the data descriptor field allows implementations to write archives
without needing to seek in the output file. This allows Zip files to
be written to pipes for example, sending an image over the network
using netcat or ssh.

\begin{figure}[tb]
  \begin{center}
  \mbox{\epsfxsize=0.8\columnwidth \epsffile{zip_structure.eps}}
  \caption{The basic structure of a Zip archive}
  \label{zip_structure}
  \end{center}
\end{figure}

It is important to note that AFF4 only requires that the underlying storage mechanism
be capable of storing multiple named segments of data. Although our
AFF4 implementation will use the ZIP64 file format as an underlying storage
mechanism, our system also supports legacy AFF1 volumes as well as
segments that are stored in other storage systems---for example,
Amazon.com's Simple Storage System (S3)\cite{s2-aws-home-page-money}.

\subsubsection{Unused ZIP64 Features}

There are a number of aspects in the Zip file format specifications
which we do not utilize. 
\begin{itemize}
\item We ignore Zip64's built-in support for splitting
archives into multiple ZIP files. Instead, our implementation treats
each volume as complete and stand-alone ZIP file. The AFF4
implementation then considers the segments contained within as 
belonging to a larger collection. The reason behind this decision is that
often it is needed to add new volumes to an existing set without
modifying the existing set at all. Zip64's built-in support for
splitting an archive renders each individual volume in the set
unusable unless all are available. 

\item Although Zip64 also defines encryption and authentication
extensions, we do not use them due to the
restrictions imposed on their use and because they lack of functionality that is
important for forensic user. Instead, we use AFF1's digital signature
facilities for integrity and non-repudiation, and we introduce a new  stream based
encryption scheme for ensuring data privacy (Section~\ref{crypted_stream}).

\end{itemize}

\subsubsection{A Forensically Sound Strategy For Updating ZIP64 Files}

In many forensic applications, individual segments within the volume
need to be updated. To this end, Garfinkel implemented several
``update'' functions in AFF1\cite{garfinkel:aff}. 

There were two problems with the original AFF updating approach:
\begin{itemize}
\item Updates can cause the volume to
become fragmented as holes appear within the middle of the volume, and
updated segments are added at the end.
\item Updates could be performed silently, raising questions of
  forensic soundness.
\end{itemize}

In AFF4, when a new segment is added to an existing volume, the new
segment header overwrites the volume central directory, and the
central directory should be rewritten including the new entry
immediately after the last file in the archive. If it is not possible
to overwrite the file, the new segment can be written at the end,
followed by an updated copy of the central directory (in that case the
old central directory remains in the file). This is illustrated in
Figure \ref{zip_structure}. Implementations may delay
writing the central directory until the file is closed; should the
system crash before the central directory is written, it can be
recovered by scanning the entire ZIP file.

A segment within the archive is considered to be updated when a new
segment is added to the archive with the same name and a later
timestamp. Implementations satisfy read requests for duplicate
segment names with those segments with the later timestamp, or if the
two timestamps are the same, the segment which appears at a later
offset in the volume. This effectively allows for a segment to be
updated without leading to internal volume fragmentation.

The fact that the older version of all updated segments remains intact
within each volume can be used for auditing, or to allow a rollback to
previous versions of the segment. We have created a new tool,
\texttt{afrepack}, which can repack, split and manage AFF4 volumes by
creating new volumes and redistributing segments among them.

\section{Standard AFF4 Streams}
In this section we discuss specific stream types that we have created
in our initial AFF4 implementation. 

\subsection{The Image Stream}
\label{image_stream}
The AFF4 \emph{Image} stream contains a single read-only forensic data
set. For example, this stream might contains a hard
disk image, a memory image or a network capture (in PCAP
format). Random read access into the stream is provided with
AFFLIB. (AFFLIB also provided the ability to randomly \emph{write}
into the evidence stream for sanitization purposes, although this
feature may be removed at a later date.) 

Image streams must have at several attributes. First, they must have
\texttt{type=image} to indicate that they are image streams. Streams
must have a \texttt{size} attribute (specified in 8-bit bytes) to
indicate the last addressable byte in the stream. Streams can have
holes in them; in this case, attempts to read data from within the
holes can generate an error or return NULLs.

The image itself is stored in \emph{pages} within the AFF4 file. Pages are
segments named as an 8 digit, zero padded decimal integer
representation of their chunk id (e.g. ``default/00000032''). Image
streams must specify the \texttt{page\_size} attribute, as the number of
image bytes each chunk contains. Pages may be compressed at the
page level, or may be broken up into multiple chunks (typically 8KB),
and individually compressed. In this case the beginning of the page
segment will contain an offset table indicating the location of each
compression chunk. This is similar to the technique used by the Expert
Witness file format used by EnCase\cite{encase-3.0} and implemented by
the open source libewf\cite{libewf} package.

Pages can optionally be individually hashed, allowing for detection
of modifications. The attribute \texttt{hash\_type} specifies the type of
hash used (implementations will typically support md5,sha1 and sha256, but can
define others). The hashes are stored in any number of segments within
the stream, of arbitrary name. The hash segment names is provided by
using the \texttt{hash\_name} stream attribute.

Hashes are calculated in an identical way to that defined by AFF1's
crypto layer\cite{garfinkel:affcrypto}---that is, the full name of the
segment (including stream component) is followed by a single NULL
byte, followed by the (uncompressed) content of the chunk. The
inclusion of the segment name ensure that even identical chunks in
content will have a different hash and prevents reordering attacks.

Each hash segment is a list of a chunk id encoded in 32 bit
little endian integer format followed by the hash (the length of which
depends on the hash itself). This approach is illustrated in Figure
\ref{hash_index}. 

Hash segments can contain any number of chunk hashes in them. It is
recommended that a hash segment be created when a volume is closed to
cover all the chunks within that volume. This allows for hashes to be
verified even if some volumes are missing. The AFFLIB implementation
may update a hash of a chunk previously set by an earlier hash
segment---this would be required when a chunk is updated. The hash
segment can then be signed to assure the integrity of the updated segment.

\begin{figure}[tb]
  \begin{center}
  \mbox{\epsfxsize=0.6\columnwidth \epsffile{hash.eps}}
  \caption{The structure of a hashing segment. The attributes {\em
  hash} and {\em hash\_type} declare the hash segment. The segment
  contains a list of integer encoded chunk id followed by an unencoded
  hash for each chunk.}
  \label{hash_index}
  \end{center}
\end{figure}

This design differs from AFF1, which creates a separate segment for
each hash. AFF4 uses headers that are more verbose than the AFF
segment headers. Therefore, creating a new segment for each hash value
would result in too much overhead: Packing the hashes into a few
segments it is possible to reduce this overhead to a minimum.

\subsection{The Map Stream}
\label{map_stream}
Linear transformations of data are commonplace in forensic
analysis. For example, a file is often simply a collection of bytes
drawn from an image, while a TCP/IP stream is simply a collection of
payloads from selected network packets. 
Sometimes the same data may be
viewed in a number of ways---for example a Virtual Address Space is a
mapping of the Physical Address Space through a page table
transformation.  Zero Storage Carving \cite{Meijer2006} is a way of
specifying carved files in terms of a sequence of blocks taken from
the image; Cohen extended this concept to an arbitrary
mapping function\cite{Cohen2007} . 

AFF4 defines a common way in
which the underlying storage system can apply transformations to
existing data to produce transformed data. This transformation is
performed using a \emph{map stream}.

The Map stream defines a mapping function, as a piecewise continuous
linear transform of one or more {\em target} streams to produce a new
{\em stream}. Target streams are specified by using ``target''
attribute, which may be used many times to provide a list of targets.

The mapping function file consists of a series of lines, each
containing three 64 bit integers separated by comma, and encoded using
decimal notation. The integers represent stream offset, target offset
and target number. Target numbers are an index to the list of targets
given in the attribute section (list index starts at zero for the
first entry).

Denoting the stream offset by $x$, and the target offset by $y$, the
Map specifies a set of points $(X_i,Y_i,T_i)$. Read requests for a
byte at a mapped stream offset $x$ can then be satisfied by reading a
byte from target $T_i$ at offset $y$ given by:
\begin{eqnarray}
y = (x - X_i) + Y_i & &
\forall x \in \left [X_i, X_i+1 \right )
\end{eqnarray}

For example, consider the following map:
\begin{verbatim*}
0,0,0
4096,10000,0
8192,5000,0
\end{verbatim*}

To read this stream we satisfy read requests of offsets between 0 and
4095 in the stream from offset 0 in target 0. Requests for bytes
between 4096 and 8191 are fetched from target 0 at offset
10000. Finally bytes after 8192 (until the specified size of the
stream) are fetched from offset 5000 in target 0.

In order to efficiently express periodic maps such as those found in
RAID arrays, the Map stream may be provided with two optional
parameters a {\em target\_period} ($T_p$), and {\em stream\_period}
($S_p$). If specified, the above relation becomes:
\begin{eqnarray*}
p &:=& floor\left (\frac{x}{S_p} \right) \\
x' &:=& mod(x ,S_p)  \\   \label{eq:no1}
y &:=& (x'-X_i) + Y_i + p \times T_p
\end{eqnarray*}

Where $mod$ is the modulus function and $floor$ signifies integer
division. For example consider Figure~\ref{map}, which corresponds to a 3
disk RAID-5 array.

\begin{figure}
\begin{Verbatim}[frame=single]
stream_period=393216
target_period=196608

0,0,1
65536,0,0
131072,65536,2
196608,65536,1
262144,131072,0
327680,131072,2
\end{Verbatim}
\caption{A Map stream that corresponds to a 3 disk RAID-5 array}\label{ref}
\end{figure}

\subsection{The Overlay Stream}
The Overlay Stream is a variant of the {\em Image} stream that
provides a layer of indirection between requests made for specific
bytes (or chunks or bevys) and the actual location where the requested
information is stored. The main utility of the Overlay stream is in
providing transparent access to legacy forensic formats, such as EWF.

To access these legacy file formats we only need to have an index of
each chunk offset, and then directly use the overlayed files.  The {\em
ewf2aff} tool provides such an overlay feature for EWF volumes by
employing libewf to scan the internal data structures.

Overlay stream require that a \texttt{chunk\_size} attribute be specified. In
addition the {\em compression\_type} attribute may be specified. The
attribute should contain an integer corresponding to a suitable Zip
compression scheme (e.g. DEFLATE). If not specified it is assumed
that no compression is used.

The ``target'' stream attribute may be specified multiple times and
refer to external files (See section \ref{target_reference}). There
can be a number of overlay segments named ``stream name/overlay.00''
which are all merged to a single overlay. If more than one overlay is
present, the attribute {\em overlays} MUST be specify how many.

The overlay segment contains a series of lines with comma delimited,
decimal encoded, 64 bit integers representing chunk number, target
offset, target length and target number. Target number is an index
into the targets defined in the attributes (count starts with zero).
Chunks are then fetched from the specified target and decompressed if
necessary.

It is possible to refer to the current file by providing a target of
``file://.''. This can be useful for modifying an existing EWF file to
be a AFF4 file. Since Zip files are normally read from the end of the
file, but EWF are both read from the front of the file, it may
be possible to append a AFF4 Central Directory to the end of an EWF file without
interfering with the overlayed file. In this case the converted file
can still be used as an EWF file without change, but could also be
used as an AFF4 file.

In image streams some chunks may compress very well. In that case, the
overheads introduced by the Zip File Header for each chunk in the
Image stream may become unacceptably high. In that case its possible
to use an {\em Overlay stream} to coalesce chunks into larger
segments. This approach is illustrated in Figure \ref{overlay}.

\begin{figure}[tb]
  \begin{center}
  \mbox{\epsfxsize=0.6\columnwidth \epsffile{overlay.eps}}
  \caption{The use of an Overlay Stream to overlay an image with
coalesced chunks.}
  \label{overlay}
  \end{center}
\end{figure}

In the above figure a segment is written to the AFF4 volume which
contains a number of chunks back to back. An overlay stream then uses
direct references to each compressed chunk within the coalesced
segment. The target is then specified as the segment name. Although
this technique is valid, it is generally discouraged since it breaks
the appealing simplicity of the standard Image Stream.  It is no
longer possible to simply unzip all the chunks and concatenate them
together to recover the original image.



\subsection{Encrypted Streams}
Encryption is an important property in an evidence file format. In
particular, multiple streams may be present in the file set, and often
different access levels are desired. For example, for evidence set
containing both network captures and disk images it may be desirable
to limit access to streams based on legal authorizations.

Although the Zip64 standard specifies encryption, it is not suitable
for our purposes since it encrypts each segment separately, and does
not specify a sufficiently flexible scheme (e.g. support for PKI, PGP
keys). Segment based encryption may lead to information leakage when
segments are compressed, as the uncompressed size of the segment may
be deduced.

AFF4 therefore introduces a new encryption scheme, the Encrypted
Stream.  The Encrypted Stream is a stream that encapsulates other
streams on a bevy-by-bevy basis. Each bevy transparently encrypted upon
writing and decrypted on reading. The choice of algorithm and mode is declared in the
Stream's \texttt{scheme} attribute. 

The Encrypted Stream may contain any data at all. It is useful
however, to store an actual AFF4 volume within the Encrypted
stream. This provides block level encryption for the contained AFF4
volume. It is recommended that Encrypted Streams containing a AFF
volume set an attribute (\texttt{content-type=application/x-aff-volume})
to allow implementations to automatically recognize volumes. This
approach is illustrated in Figure~\ref{crypted_fif}.

\begin{figure}[tb]
  \begin{center}
  \mbox{\epsfxsize=0.8\columnwidth \epsffile{crypted.eps}}
  \caption{Embedding an encrypted AFF4 volume within an Encrypted Stream.}
  \label{crypted_fif}
  \end{center}
\end{figure}

The result is that a number of AFF4 volumes are used as {\em Container
Volumes} to provide storage for Encrypted Streams. The main {\em
Embedded Volume}, which actually contains data is stored within the
Encrypted Stream, effectively distributed throughout the container
volumes. Note that the outer Volume may contain several Encrypted
Streams and therefore contain multiple AFF4 Encrypted 
Volumes. Container Volumes may contain non encrypted streams as well,
and may implement different encryption schemes and keys for each
Encrypted scheme. This effectively allows arbitrary access policies to
be implemented as only volumes which can be accessed can be merged.

\subsection{Encryption Schemes}
We specify a number of encryption schemes, but extra schemes may be
defined by the implementation.

\subsubsection{Null Encryption}
This scheme ({\em scheme=null}) specifies no encryption at all. This
may be useful for testing an implementation, but obviously provides
no real security.

\subsubsection{AES-SHA-PSK}
This scheme ({\em scheme=aes-sha-psk}) uses AES for block level
encryption, with the following SHA based ESSIV scheme, and Pre-Shared
Key (passphrase).

When creating a new volume, a master key is generated using the first
128 bits of a SHA1 hash of the Pre-Shared-Key appended to a 64 bit
random salt:
\begin{eqnarray}
Key_{Master} = \left | SHA1(salt + PSK) \right | _{128}
\end{eqnarray}

For each chunk, the chunk IV is obtained by taking the first 128 bits
of the SHA1 hash of the chunk id encoded as a 32 bit little endian
integer, followed by the master key:
\begin{eqnarray}
IV = \left | SHA1(chunk\_id + Key_{Master}) \right | _{128}
\end{eqnarray}

The block is then encrypted or decrypted using AES with this IV.

Implementations MUST specify the salt as a stream attribute named {\em
salt} in base 64 encoding. Implementations are free to specify any API
for passing the Pre-Shared Key, for example the PSK can be specified
in a configuration file, typed in at the terminal, passed in using
process environment variables, or encoded within the AFF4 filename
itself.

\subsection{Private Streams}
Often forensic software needs to store internal state, such as the
current state of the GUI, or internal data structures which may be
required for caching. Usually, software create additional cache files
to maintain this information. It is advantageous to be able to store
these in the evidence file itself. The evidence file can then be
re-opened by the software at a later stage and private data can be
retrieved and used directly. Such private application data can be
stored in a Private stream. There is no specification of what can be
stored in the private stream, and applications with private streams
should be able to store arbitrary segments in any format at all.

It is recommended that applications name their private segment
sufficiently accurately so as not to be confused with other
applications, or different versions of the same application
(e.g. ``pyflag/0.87/cache/'').



\section{Implementing AFF4 in AFFLIB}
Although there are numerous ZIP implementations available
today, we have created our own implementation using a combination of
available open source technology and our own novel contributions. We
are in the process of integrating this code into the AFFLIB library
and performing extensive regression testing. 

There are many reasons to develop our own ZIP64 implementation for
AFF4:

\begin{itemize}
\item Most ZIP implementations do not implement the ZIP64
  extensions. These extensions are required to support Evidence
  Volumes larger than 2GB.
\item Simple Zip implementations might rescan the
Central Directory for each segment request. Since in practice there
can be a large number of segments in a volume, it is advisable to have
a ZIP64 implementation that is optimized to storing thousands (or even
hundreds of thousands) of segments in an efficient data structure. 
\item While the ZIP specification duplicates data found in the Central
  Directory entry in each File Header (such as filename, size, CRC
  etc), many implementations that we have examined only populate this
  information in one of these places. In the interest of robustness,
  we wanted to assure that data stored in both locations would be
  populated to allow recovery of at least \emph{some} evidence that
  might exist in damaged volumes. If central directory is lost, it is
  possible to scan through the zip file and repair the central
  directory from the File headers.

\item Our implementation supports simultaneous access by multiple
  readers and writers. Our system allows concurrency by locking the
  file and interleaving write operations. Managing simultaneous
  writers is simplified through our update strategy which only writes
  new segments to the file, never modifying old segments. 
\item AFF4 ``properties'' have specific semantics above and beyond
  normal segments. For example, AFF4 calls for all of the property
  segments associated with a file to be merged together. Furthermore,
  when updating the properties in a multi-volume Set, it is useful to
  write the same properties to each volume to ensure at least some of
the stream can be used even if some volumes are lost. Such
functionality is not present in standard ZIP64 implementations.
\end{itemize}

\section{Usage Scenarios}
In this section we describe how a FIF file format may be used in
various situations. Many of those can be solved using other forensic
file formats, but often in a more awkward way.

\subsubsection{Rapidly converting a set of DD images}
Many hardware devices are available to acquire hard disks in the
fields. These often produce a set of uncompressed images split at a
certain size. If the images are split at exactly the same size its
possible to create a FIF archive consisting of a single Overlay
Stream, with a chunk\_size set to the size of each of the images. The
images can then be listed in order as external targets (e.g. {\em
target=file://disk1.dd}). This overlay can be created instantly
without needing to re-compress any of the images. In addition, digital
signatures can be generated for the whole stream or for each segment.

\subsubsection{Rapid conversion of EWF file sets}
Much existing evidence has been acquired using the EWF file
format. This format, similarly, consists of a series of chunks which
may be compressed, with interdispersed indexes. A tool such as
``ewf2fif'' is able to utilize the libewf library to create an Overlay
stream of the original image without the need to re-compress it. This
conversion is very quick and the resulting FIF file is very small -
each volume in the EWF file set is referenced through an external
target reference, and the Overlay Stream is built as an index of each
chunk location.

\subsubsection{Acquisition of RAID disks}
Often disks in a system are grouped into RAID devices, commonly RAID-5
or RAID-0. Previously, if disks were acquired independently, they
would need to be analysed using a tool which was able to reassemble
RAID devices.

With the FIF format, each of the disks can be acquired as a separate
Image Stream. Finally a tool such as PyFlag may be used to deduce the
RAID map, which can be appended to the FIF file as a Map Stream. This
Map Stream can then be opened by any tool without the tool needing to
have explicit support for RAID reassembling.

\subsubsection{Cryptographic management of evidence}
A FIF archive may hold multiple encrypted volumes, each in its own
Encrypted Stream. Each of those streams is encrypted using a different
master key, and therefore can have different passphrases, and can be
assigned to different users by encrypting the master key with
different X509 certificates. Its is also possible for users to create
non-encrypted volumes within the FIF file.

This can be used to enforce access controls in line with current
legislative requirements. For example, within the same investigation
different material is often obtained under different warrants
(e.g. wiretap authorizations are different from search
warrants). Therefore, different investigators and analysts need
different access to the different streams. However, the analysts may
still store the results of their analysis in an unencrypted form, or
assign others permissions to decrypt their analysis results, without
providing access to the underlying data. 

This can be used in sharing meta data (e.g. Map Streams of files of
interest) between analysts, without needing to provide access to the
underlying data.

\subsubsection{Forensic Application State}
Often forensic applications need to store files other than the
evidence itself - for example, they might need to store internal data,
annotations, keyword hits etc. Currently these applications store the
data in a proprietary external file or database. This makes it
difficult to archive because the evidence itself may become separated
from the case file.

It is advantageous for these applications to store their state within
the evidence file itself using a Private Stream. Then when opening the
evidence file again, the results of their analysis will become
available.

\subsubsection{Merging of evidence set}
In this common usage case, Bob, goes on site to image one hard disk,
while Alice independently images a second hard disk. After returning
to the office they both realise their images belong to the same
set. They merge their images, by updating the UUIDs on one of the
volume set to now belong to the other volume set - the merge is almost
instantaneous as they only need to write an extra ``properties'' file
on the merged volumes.

\section{Conclusion and Future Work}

\bibliography{IEEEabrv,paper}
\end{document}



%%%%%%%%%%%%%%%%%%%%%%%%%%%%%%%%%%%%%%%%%%%%%%%%%%%%%%%%%%%%%%%%

